\addcontentsline{toc}{section}{СПИСОК ИСПОЛЬЗОВАННЫХ ИСТОЧНИКОВ}

\begin{thebibliography}{9}
	
	\bibitem{chinnam2023} Чиннам, Р. Полный курс веб-разработки: Python, Django, JavaScript, Vue.js / Р. Чиннам. – Москва: Бомбора, 2023. – 896 с. – ISBN 978-5-04-170621-4. – Текст: непосредственный.
	
	\bibitem{grinberg2018} Гринберг, М. Flask. Разработка веб-приложений на Python / М. Гринберг. – СПб.: Питер, 2018. – 352 с. – ISBN 978-5-4461-0400-3.
	
	\bibitem{ramalho2021} Рамальо, Л. Python. К вершинам мастерства / Л. Рамальо. – СПб.: Питер, 2021. – 768 с. – ISBN 978-5-4461-1349-4.
	
	\bibitem{lutz2019} Лутц, М. Изучаем Python / М. Лутц. – СПб.: Символ-Плюс, 2019. – 1648 с. – ISBN 978-5-93286-159-5.
	
	\bibitem{beazley2020} Бизли, Д. Python. Подробный справочник / Д. Бизли. – СПб.: Символ-Плюс, 2020. – 864 с. – ISBN 978-5-93286-211-0.
	
	\bibitem{tanenbaum2022} Таненбаум, Э. Современные операционные системы / Э. Таненбаум. – СПб.: Питер, 2022. – 1120 с. – ISBN 978-5-4461-1156-8.
	
	\bibitem{django4book} Дронов, В. Django 4. Практика создания веб-сайтов на Python / В. Дронов. – СПб.: БХВ-Петербург, 2022. – 592 с. – ISBN 978-5-9775-4065-1.
	
	\bibitem{restfulapi} Ричардсон, Л. RESTful Web API / Л. Ричардсон. – СПб.: Питер, 2016. – 448 с. – ISBN 978-5-496-02056-0.
	
	\bibitem{postgresqlpro} Клайн, К. PostgreSQL для профессионалов / К. Клайн. – М.: ДМК Пресс, 2021. – 480 с. – ISBN 978-5-97060-881-8.
	
	\bibitem{vuejs3} Гребач, Е. Vue.js 3. Разработка фронтенд-приложений / Е. Гребач. – СПб.: Питер, 2022. – 352 с. – ISBN 978-5-4461-1703-4.
	
	\bibitem{htmlcss} Дакетт, Д. HTML и CSS. Разработка и дизайн веб-сайтов / Д. Дакетт. – М.: Эксмо, 2021. – 480 с. – ISBN 978-5-04-113690-2.
	
	\bibitem{python_testing} Перельройзен, А. Тестирование Python. Руководство / А. Перельройзен. – СПб.: БХВ-Петербург, 2020. – 368 с. – ISBN 978-5-9775-4064-4.
	
	\bibitem{api_design} Амундсен, М. Дизайн RESTful API / М. Амундсен. – М.: ДМК Пресс, 2019. – 232 с. – ISBN 978-5-97060-710-1.
	
	\bibitem{sql_expert} Бьюли, А. SQL. Сборник рецептов / А. Бьюли. – СПб.: Символ-Плюс, 2020. – 656 с. – ISBN 978-5-93286-223-3.
	
	\bibitem{web_security} Зафер, Х. Веб-безопасность для разработчиков / Х. Зафер. – СПб.: Питер, 2021. – 416 с. – ISBN 978-5-4461-1463-7.
	
	\bibitem{python_cookbook} Бизли, Д. Python. Книга рецептов / Д. Бизли, Б. Джонс. – СПб.: Символ-Плюс, 2020. – 704 с. – ISBN 978-5-93286-222-6.
	
	\bibitem{django_rest} Агравал, В. Django REST Framework / В. Агравал. – М.: ДМК Пресс, 2021. – 280 с. – ISBN 978-5-97060-882-5.
	
	\bibitem{postgresql_optimization} Каммингс, Д. PostgreSQL. Оптимизация запросов / Д. Каммингс. – СПб.: БХВ-Петербург, 2022. – 448 с. – ISBN 978-5-9775-4075-0.
	
	\bibitem{vuejs_components} Марина, К. Компоненты Vue.js / К. Марина. – СПб.: Питер, 2021. – 304 с. – ISBN 978-5-4461-1540-5.
	
	\bibitem{css_mastery} Бадд, Э. Мастерство CSS / Э. Бадд. – М.: Эксмо, 2022. – 416 с. – ISBN 978-5-04-119944-0.
	
	\bibitem{pytest_book} Окабэ, Х. Pytest. Полное руководство / Х. Окабэ. – СПб.: БХВ-Петербург, 2023. – 320 с. – ISBN 978-5-9775-4085-9.
	
	\bibitem{api_security} Вандерсток, Н. Безопасность API / Н. Вандерсток. – М.: ДМК Пресс, 2022. – 264 с. – ISBN 978-5-97060-883-2.
	
	\bibitem{sql_performance} Петков, Г. SQL. Оптимизация производительности / Г. Петков. – СПб.: Питер, 2021. – 384 с. – ISBN 978-5-4461-1498-9.
	
	\bibitem{web_development} Фримен, Э. Современная веб-разработка / Э. Фримен. – СПб.: Питер, 2022. – 768 с. – ISBN 978-5-4461-1688-4.
	
	\bibitem{python_algorithms} Миллер, Б. Алгоритмы на Python / Б. Миллер, Д. Ранум. – СПб.: Символ-Плюс, 2021. – 480 с. – ISBN 978-5-93286-224-0.
	
	\bibitem{django_channels} Лоу, Э. Django Channels в действии / Э. Лоу. – М.: ДМК Пресс, 2023. – 216 с. – ISBN 978-5-97060-884-9.
	
	\bibitem{rest_practice} Филдс, Д. REST на практике / Д. Филдс. – СПб.: Питер, 2020. – 512 с. – ISBN 978-5-4461-1338-8.
	
	\bibitem{postgresql_admin} Риггс, С. PostgreSQL. Администрирование / С. Риггс. – СПб.: БХВ-Петербург, 2021. – 592 с. – ISBN 978-5-9775-4070-5.
	
	\bibitem{vuex_book} Хертинг, А. Vuex для профессионалов / А. Хертинг. – СПб.: Питер, 2022. – 288 с. – ISBN 978-5-4461-1723-2.
	
	\bibitem{html5_book} Роббинс, Д. HTML5. Карманный справочник / Д. Роббинс. – М.: Эксмо, 2021. – 192 с. – ISBN 978-5-04-119945-7.
	
	\bibitem{python_web} Гусев, И. Python и веб-разработка / И. Гусев. – СПб.: БХВ-Петербург, 2022. – 448 с. – ISBN 978-5-9775-4078-1.
	
	\bibitem{django_security} Хилл, Д. Безопасность Django / Д. Хилл. – М.: ДМК Пресс, 2023. – 184 с. – ISBN 978-5-97060-885-6.
	
	\bibitem{api_design_patterns} Джорджеску, Д. Паттерны проектирования API / Д. Джорджеску. – СПб.: Питер, 2021. – 352 с. – ISBN 978-5-4461-1502-3.
	
	\bibitem{sql_advanced} Молинаро, Э. SQL. Продвинутый уровень / Э. Молинаро. – СПб.: Символ-Плюс, 2022. – 528 с. – ISBN 978-5-93286-225-7.
	
	\bibitem{web_components} Жиру, П. Веб-компоненты в действии / П. Жиру. – СПб.: Питер, 2023. – 336 с. – ISBN 978-5-4461-1763-8.
	
	\bibitem{python_concurrency} Паласиос, М. Параллельное программирование на Python / М. Паласиос. – СПб.: БХВ-Петербург, 2021. – 384 с. – ISBN 978-5-9775-4072-9.
	
	\bibitem{django_performance} Картер, Н. Оптимизация Django / Н. Картер. – М.: ДМК Пресс, 2022. – 248 с. – ISBN 978-5-97060-886-3.
	
	\bibitem{graphql_book} Бэнкс, А. GraphQL в действии / А. Бэнкс. – СПб.: Питер, 2021. – 320 с. – ISBN 978-5-4461-1520-7.
	
	\bibitem{postgresql_programming} Фонг, Д. Программирование на PostgreSQL / Д. Фонг. – СПб.: Символ-Плюс, 2023. – 624 с. – ISBN 978-5-93286-226-4.
	
	\bibitem{vue_testing} Эдвардс, Р. Тестирование Vue.js приложений / Р. Эдвардс. – СПб.: Питер, 2022. – 256 с. – ISBN 978-5-4461-1743-0.
	
	\bibitem{css_grid} Гарднер, Д. CSS Grid в действии / Д. Гарднер. – М.: Эксмо, 2022. – 288 с. – ISBN 978-5-04-119946-4.
	
	\bibitem{python_architecture} Аркуш, П. Архитектура Python-приложений / П. Аркуш. – СПб.: БХВ-Петербург, 2023. – 416 с. – ISBN 978-5-9775-4088-0.
	
	\bibitem{django_tests} Гарсия, Х. Тестирование Django-приложений / Х. Гарсия. – М.: ДМК Пресс, 2021. – 200 с. – ISBN 978-5-97060-887-0.
	
	\bibitem{microservices_api} Ньюмен, С. Микросервисы и API / С. Ньюмен. – СПб.: Питер, 2022. – 448 с. – ISBN 978-5-4461-1683-9.
	
	\bibitem{sql_design} Карасик, Д. Проектирование SQL-баз данных / Д. Карасик. – СПб.: Символ-Плюс, 2021. – 480 с. – ISBN 978-5-93286-227-1.
	
	\bibitem{web_accessibility} Макдональд, М. Веб-доступность / М. Макдональд. – СПб.: Питер, 2023. – 384 с. – ISBN 978-5-4461-1783-6.
	
	\bibitem{python_web_scraping} Митчелл, Р. Веб-скрейпинг на Python / Р. Митчелл. – СПб.: БХВ-Петербург, 2022. – 352 с. – ISBN 978-5-9775-4079-8.
	
	\bibitem{django_deployment} Сингх, Р. Деплой Django-приложений / Р. Сингх. – М.: ДМК Пресс, 2022. – 232 с. – ISBN 978-5-97060-888-7.
	
	\bibitem{rest_security} Харди, М. Безопасность REST API / М. Харди. – СПб.: Питер, 2021. – 304 с. – ISBN 978-5-4461-1542-9.
	
	\bibitem{postgresql_replication} Гамма, Э. Репликация в PostgreSQL / Э. Гамма. – СПб.: Символ-Плюс, 2023. – 576 с. – ISBN 978-5-93286-228-8.
	
	\bibitem{vue_performance} Ллойд, Д. Оптимизация Vue.js приложений / Д. Ллойд. – СПб.: Питер, 2022. – 272 с. – ISBN 978-5-4461-1763-8.
	
	\bibitem{html_css_reference} Макфарланд, Д. HTML и CSS. Полный справочник / Д. Макфарланд. – М.: Эксмо, 2023. – 1088 с. – ISBN 978-5-04-119947-1.
	
	\bibitem{python_networking} Роуз, Б. Сетевые приложения на Python / Б. Роуз. – СПб.: БХВ-Петербург, 2021. – 400 с. – ISBN 978-5-9775-4080-4.
	
	\bibitem{django_orm} Фернандес, Л. Django ORM в глубине / Л. Фернандес. – М.: ДМК Пресс, 2023. – 264 с. – ISBN 978-5-97060-889-4.
	
	\bibitem{api_monitoring} Кригер, С. Мониторинг API / С. Кригер. – СПб.: Питер, 2022. – 336 с. – ISBN 978-5-4461-1788-1.
	
	\bibitem{sql_antipatterns} Карп, Б. Антипаттерны SQL / Б. Карп. – СПб.: Символ-Плюс, 2021. – 432 с. – ISBN 978-5-93286-229-5.
	
	\bibitem{web_performance} Григсби, Д. Веб-производительность / Д. Григсби. – СПб.: Питер, 2023. – 416 с. – ISBN 978-5-4461-1883-3.
\end{thebibliography}
