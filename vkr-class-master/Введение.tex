\section*{ВВЕДЕНИЕ}
\addcontentsline{toc}{section}{ВВЕДЕНИЕ}

Фриланс задал направление развития удалённого взаимодействия сотрудников и руководителя. Так же такая идея взаимодействия заказчика и исполнителя оказалась вполне удобной, так как сотрудник может работать из любой точки мира и регламентировать самому своё рабочее время, так как главным при таком взаимодействии является результат. 

Современные заказчики, видя, как развиваются информационные технологии, пытаются использовать такую идею взаимодйствия выгодно для своего бизнеса, поэтому оставляют свои заказы на проекты в сфере информационных технологий на подобных биржах. С их помощью заказчи может за определённую сумму отобрать необходимого кандидата и получить полноценный результат в определённый срок.

Главной задачей профессионально построенного сервиса является превращение посетителя, зашедшего на сайт, в пользователя, который будет активно пользоваться предоставляемыми услугами сайта.

\emph{Цель настоящей работы} – разработка web сервиса фриланс биржи для удобного взаимодействия заказчиков и исполнителей. Для достижения поставленной цели необходимо решить \emph{следующие задачи:}
\begin{itemize}
\item провести анализ предметной области;
\item разработать концептуальную модель web-сайта;
\item спроектировать web-сайт;
\item реализовать сайт средствами web-технологий.
\end{itemize}

\emph{Структура и объем работы.} Отчет состоит из введения, 4 разделов основной части, заключения, списка использованных источников, 2 приложений. Текст выпускной квалификационной работы равен \formbytotal{lastpage}{страниц}{е}{ам}{ам}.

\emph{Во введении} сформулирована цель работы, поставлены задачи разработки, описана структура работы, приведено краткое содержание каждого из разделов.

\emph{В первом разделе} на стадии описания технической характеристики предметной области приводится сбор информации о подобных уже существующих сервисах, которые предоставляют услуги фриланса.

\emph{Во втором разделе} на стадии технического задания приводятся требования к разрабатываемому сервису.

\emph{В третьем разделе} на стадии технического проектирования представлены проектные решения для web-сайта.

\emph{В четвертом разделе} приводится список классов и их методов, использованных при разработке сайта, производится тестирование разработанного сайта.

В заключении излагаются основные результаты работы, полученные в ходе разработки.

В приложении А представлен графический материал.
В приложении Б представлены фрагменты исходного кода. 
