\section{Анализ предметной области}
\subsection{Фриланс биржи. История возникновения фриланса}

Термин фриланс (от англ. freelance) возник от слова фрилансер, что в переводе с английского переводится как внештатный сотрудник. Ввод данного термина приписывают британскому поэту и историку Вальтеру Скотту -- он его использовал ещё в романе "Айвенго" 1819 года выхода, где дословно переводя с английского термин обозначается "вольный копейщик".

Но точный период зарождения такого понятия, как фриланс, назвать сложно, но точно можно сказать, что в 90-х годах появлялись первые удалённые работники. Выделить можно один из первых сервисов Guru.com, основанный в 1998 году. Позже, в 1999 году, появилась биржа ELance.

В 2000-х годах, в виду массового развития высокоскоростного интернета, произошёл бум фриланс-бирж. Так же данному явлению способствовали появление удобных платёжных систем и рост аутсорсинга - западные компании нанимали дешёвых специалистов из менее развитых стран. Яркими представителями этого периода можно считать такие биржи, как oDesk, Freelancer.com и Fiverr. Ещё можно выделить, что в этот период началась активная борьба с мошенниками - внедрение безопасных платежей и системы отзывов.

В 2010-х годах уже обособились гиганты рынка - крупные биржи закрекомендовались основными, появлялись локальные биржи в различных странах. Но так же чем интересен данный период - это появление узкоспециализированных бирж по причине перегрузки крупных платформ. 

\subsection{Как устроены фриланс биржи}

Фриланс-биржи — это сложные цифровые платформы, соединяющие заказчиков и исполнителей через продуманную систему взаимодействий. В их основе лежит четкое разделение сущностей: пользователи (заказчики, фрилансеры, модераторы), проекты с описанием задач, заявки от исполнителей, портфолио работ и финансовые транзакции. Каждый элемент системы обладает своими атрибутами и связями — например, проект включает бюджет, сроки и статус, а заявка содержит ценовое предложение и комментарий фрилансера. Ключевые бизнес-процессы построены вокруг публикации заказов, подачи откликов, безопасных платежей через Escrow-системы и механизмов обратной связи, формирующих репутацию участников.

С технической точки зрения фриланс-биржи используют клиент-серверную архитектуру, где фронтенд (веб и мобильные приложения) взаимодействует с бэкендом через API. Серверная часть включает базы данных для хранения информации, платежные шлюзы для обработки транзакций и сервисы уведомлений. В сложных системах применяется микросервисный подход, разделяющий функционал на модули: аутентификацию, управление проектами, поиск и платежи. Особое внимание уделяется безопасности — шифрование данных (HTTPS, JWT), верификация пользователей и механизмы арбитража для разрешения споров. Для борьбы с мошенничеством внедряются репутационные системы, анализ поведения и модерация контента.

Современные платформы активно используют анализ данных и машинное обучение. Рекомендательные системы предлагают заказчикам подходящих фрилансеров на основе истории заказов и навыков, а исполнителям — релевантные проекты. Алгоритмы выявляют подозрительную активность, например фейковые заказы или попытки обмана при оплате. Отдельное направление — оптимизация ценообразования, где ИИ анализирует рыночные тенденции и подсказывает исполнителям конкурентные ставки. В будущем эти технологии будут развиваться в сторону персонификации, автоматизации рутинных задач (например, составления ТЗ) и интеграции с Web3 — децентрализованными биржами на блокчейне.

\subsection{Пример фриланс биржи}

Как пример фриланс биржи можно взять такой веб-сервис, как FL.ru (ранее Free-lance.ru). Он является одной из старейших фриланс бирж, которая актуальна до сих пор, так как в отличии от Kwork с его фиксированными услугами, FL сохраняет свою стандартную модель взаимодействия заказчика и исполнителя в формате аукционов. 

Ключевыми особенностями данного сервиса являются аукционная система и безопасность сделок. 

Аукционную систему можно представить в виде данной схемы: заказчик публикует техническое задание, а исполнители подают свои заявки с ценовыми предложениями. Для заказчика это выгодно, так как он может получить выполнение необходимой задачи за как можно меньшую цену, но для исполнителей это может обернуться тем, что новички намеренно будут занижать цену ради первых отзывов. В данном случае у опытных специалистов открывается несколько путей: либо занижать ещё сильнее цену, либо искать другую задачу.

Безопасность сделок на данном сервисе заключается в том, что средства замораживаются до подтверждения выполнения работы. При данной системе положительные стороны есть как для заказчика, так как если работа не была выполнена - он может вернуть средства, так и для исполнителя, так как все транзакции проходят через сам сервис.

Так же при анализе сервиса FL.ru были выявлены и слабые стороны.

Первой слабой стороной можно назвать высокую комиссию - здесь она составляет 20\% с каждой сделки для обычных пользователей и 10\% для пользователей подписки PRO.

Второй стороной является то, что на данном сервисе отсутствует чёткий стандарт написания ТЗ, в связи с чем могут возникать различные споры и конфликты между заказчиком и исполнителем.